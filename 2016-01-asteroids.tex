\documentclass{article}

\usepackage[utf8]{inputenc}
\usepackage[ngerman]{babel}

\title{Asteroids}
\date{23. Juni 2016}

\begin{document}
\maketitle

\section{Gameloop}

\section{Movement}
\subsection{Homogene Koordinaten}
\subsubsection{Karthesische Koordinaten zu homogenen Koordinaten}
\begin{description}
 \item[Ortsvektoren:]
  \begin{equation}
   \left(\begin{array}{c}
    x \\
    y \\
   \end{array}\right)
   \rightarrow
   \left(\begin{array}{c}
    x \\
    y \\
    1 \\
   \end{array}\right)
  \end{equation}

 \item[Richtungsvektoren:]
  \begin{equation}
   \left(\begin{array}{c}
    x \\
    y \\
   \end{array}\right)
   \rightarrow
   \left(\begin{array}{c}
    x \\
    y \\
    0 \\
   \end{array}\right)
  \end{equation}
\end{description}

\subsubsection{Rotation und Translation}
\begin{description}
 \item[Rotationsmatriz]
  für Rotation um $\varphi$ entgegen dem Uhrzeigersinn:
  \begin{equation}
   \left(\begin{array}{ccc}
    cos(\varphi) & -sin(\varphi) & 0 \\
    sin(\varphi) & cos(\varphi) & 0 \\
    0 & 0 & 1 \\
   \end{array}\right)
  \end{equation}
  Anwendung:
  \begin{equation}
   \left(\begin{array}{ccc}
    cos(\varphi) & -sin(\varphi) & 0 \\
    sin(\varphi) & cos(\varphi) & 0 \\
    0 & 0 & 1 \\
   \end{array}\right)
   \cdot
   \left(\begin{array}{ccc}
    x \\
    y \\
    1 \\
   \end{array}\right)
   =
   \left(\begin{array}{ccc}
    x \cdot cos(\varphi) - y \cdot sin(\varphi) \\
    x \cdot sin(\varphi) + y \cdot cos(\varphi) \\
    1 \\
   \end{array}\right)
  \end{equation}
  Beispiel: Den Punkt $(2, 1)^{T}$ um $90^{\circ}$ rotieren.
  \begin{equation}
   \left(\begin{array}{ccc}
    cos(90^{\circ}) & -sin(90^{\circ}) & 0 \\
    sin(90^{\circ}) & cos(90^{\circ}) & 0 \\
    0 & 0 & 1 \\
   \end{array}\right)
   \cdot
   \left(\begin{array}{ccc}
    2 \\
    1 \\
    1 \\
   \end{array}\right)
   =
   \left(\begin{array}{ccc}
    -1 \\
    2 \\
    1 \\
   \end{array}\right)
  \end{equation}

 \item[Translation:]
  \begin{equation}
   \left(\begin{array}{ccc}
    1 & 0 & t_{x} \\
    0 & 1 & t_{y} \\
    0 & 0 & 1 \\
   \end{array}\right)
  \end{equation}
  Anwendung:
  \begin{equation}
   \left(\begin{array}{ccc}
    1 & 0 & t_{x} \\
    0 & 1 & t_{y} \\
    0 & 0 & 1 \\
   \end{array}\right)
   \cdot
   \left(\begin{array}{ccc}
    x \\
    y \\
    1 \\
   \end{array}\right)
   =
   \left(\begin{array}{ccc}
    x + t_{x} \\
    y + t_{y} \\
    1 \\
   \end{array}\right)
  \end{equation}
\end{description}
Vorteil von homogenen Koordinaten:
Eine beliebige Verkettung aufeinanderfolgender Translationen
und Rotationen kann zu einer Matrix zusammengefasst werden.


\section{Pew! Pew! Pew!}

\section{Kollisionserkennung}
\subsection{Kreis mit Kreis}

\subsection{Konvexes Polygon mit konvexem Polygon}

\subsection{konvexem Polygon mit Kreis}

\end{document}
