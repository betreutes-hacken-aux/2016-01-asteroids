\documentclass[dvips,12pt,xcolor=table]{beamer}

\usepackage[utf8]{inputenc}
\usepackage[ngerman]{babel}
\usepackage{pstricks}
\usepackage{pst-plot}
\usepackage{minted}
\usepackage{lmodern}

\begin{document}

\begin{frame}
\frametitle{Game Loop}
\framesubtitle{o}
\end{frame}


\begin{frame}
\frametitle{Bewegung}
\framesubtitle{Homogene Koordinaten}
\textbf{Karthesische Koordinaten (2D) zu homogenen Koordinaten:}
\begin{description}
 \item[Ortsvektoren:]
  \begin{equation}
   \left(\begin{array}{c}
    x \\
    y \\
   \end{array}\right)
   \rightarrow
   \left(\begin{array}{c}
    x \\
    y \\
    1 \\
   \end{array}\right)
  \end{equation}

 \item[Richtungsvektoren:]
  \begin{equation}
   \left(\begin{array}{c}
    x \\
    y \\
   \end{array}\right)
   \rightarrow
   \left(\begin{array}{c}
    x \\
    y \\
    0 \\
   \end{array}\right)
  \end{equation}
\end{description}
\end{frame}

\begin{frame}
\frametitle{Bewegung}
\framesubtitle{Homogene Koordinaten}
\textbf{Homogenen Koordinaten (2D) zu karthesischen Koordinaten: \\}
\begin{equation}
 \left(\begin{array}{c}
  x \\
  y \\
  w \\
 \end{array}\right)
 \rightarrow
 \left(\begin{array}{c}
  \frac{x}{w} \\
  \\
  \frac{y}{w} \\
 \end{array}\right)
\end{equation}
\end{frame}

\begin{frame}
\frametitle{Bewegung}
\framesubtitle{Homogene Koordinaten}
\textbf{Rotationen: \\}
Rotation um $\varphi$ entgegen dem Uhrzeigersinn:
\begin{equation}
 \left(\begin{array}{ccc}
  cos(\varphi) & -sin(\varphi) & 0 \\
  sin(\varphi) & cos(\varphi) & 0 \\
  0 & 0 & 1 \\
 \end{array}\right)
\end{equation}
\hspace{2mm}
Beispiel: Den Punkt $(2, 1)^{T}$ um $90^{\circ}$ rotieren.
\begin{equation}
 \left(\begin{array}{ccc}
  cos(90^{\circ}) & -sin(90^{\circ}) & 0 \\
  sin(90^{\circ}) & cos(90^{\circ}) & 0 \\
  0 & 0 & 1 \\
 \end{array}\right)
 \cdot
 \left(\begin{array}{ccc}
  2 \\
  1 \\
  1 \\
 \end{array}\right)
 =
 \left(\begin{array}{ccc}
  -1 \\
  2 \\
  1 \\
 \end{array}\right)
\end{equation}
\end{frame}

\begin{frame}
\frametitle{Bewegung}
\framesubtitle{Homogene Koordinaten}
\textbf{Translationen: \\}
\begin{equation}
 \left(\begin{array}{ccc}
  1 & 0 & t_{x} \\
  0 & 1 & t_{y} \\
  0 & 0 & 1 \\
 \end{array}\right)
\end{equation}
Anwendung auf einen Richtungsvektor:
\begin{equation}
 \left(\begin{array}{ccc}
  1 & 0 & t_{x} \\
  0 & 1 & t_{y} \\
  0 & 0 & 1 \\
 \end{array}\right)
 \cdot
 \left(\begin{array}{ccc}
  x \\
  y \\
  w \\
 \end{array}\right)
 =
 \left(\begin{array}{ccc}
  x + t_{x} \cdot w \\
  y + t_{y} \cdot w \\
  w \\
 \end{array}\right)
\end{equation}
Beobachtung:
Eine Translation wirkt sich nur auf Ortsvektoren,
nicht auf Richtungsvektoren aus,
da für diese $w = 0$.
\end{frame}


\begin{frame}
\frametitle{Bewegung}
\framesubtitle{OpenGL 2.x}
Zwei Matrizen:
\begin{description}
 \item[Modelview Matrix] \hfill \\
  Auslesbar mit
  \texttt{glGetFloatv(GL\_MODELVIEW\_MATRIX)}.
  Diese Matrix wird zuallerst
  auf alle übergebenen 3D-Koordinaten angewendet.
 \item[Projection Matrix] \hfill \\
  Auslesbar mit
  \texttt{glGetFloatv(GL\_PROJECTION\_MATRIX)}.
  Ihre Aufgabe besteht darin 3D-Koordinaten
  auf eine 2D-Fläche zu projizieren.
\end{description}
\end{frame}

\begin{frame}[fragile]
\frametitle{Bewegung}
\framesubtitle{OpenGL 2.x}
Auslesen mit \texttt{glGetFloatv}:\\
\vspace{0.3cm}
\textbf{C/C++:}
\begin{minted}{c}
GLfloat *m1 = new GLfloat[16];
GLfloat *m2 = new GLfloat[16];
glGetFloatv(GL_MODELVIEW_MATRIX, m1);
glGetFloatv(GL_PROJECTION_MATRIX, m2);
\end{minted}
\vspace{0.3cm}
\textbf{Python:}
\begin{minted}{python}
m1 = glGetFloatv(GL_MODELVIEW_MATRIX)
m2 = glGetFloatv(GL_PROJECTION_MATRIX)
\end{minted}
\end{frame}

\begin{frame}
\frametitle{Bewegung}
\framesubtitle{OpenGL 2.x}
Zwei Matrizen:
\begin{description}
 \item[Modelview Matrix]
 \item[Projection Matrix]
\end{description}
\texttt{glMatrixMode(mode)} legt fest,
welche Matrix aktuell bearbeitet werden soll.
Diese kann mit Befehlen wie
\texttt{glPopMatrix},
\texttt{glLoadIdentity},
\texttt{glPushMatrix},
\texttt{glMultMatrix},
\texttt{glRotated} und
\texttt{glTranslated}
bearbeitet werden.
\end{frame}


\begin{frame}
\frametitle{Pew! Pew! Pew!}
\framesubtitle{$|>$  - - - - -}
\end{frame}


\begin{frame}
\frametitle{Kollision}
\framesubtitle{Schnitt geometrischer Objekte}
\begin{center}
\psset{xunit=1cm,yunit=1cm,runit=1cm}
\begin{pspicture}(0,0)(4,4)
 \psline(0,0)(0,4)(4,4)(4,0)(0,0)
 \rput(2,2){dummy image}
\end{pspicture}

\psset{xunit=1cm,yunit=1cm,runit=1cm}
\begin{pspicture}(0,0)(4,4)
 \psline(0,0)(0,4)(4,4)(4,0)(0,0)
 \rput(2,2){dummy image}
\end{pspicture}

\end{center}
\textbf{Vorteil:} einfach zu realisieren \\
\textbf{Nachtteil:} nicht exakt
\end{frame}

\begin{frame}
\frametitle{Kollision}
\framesubtitle{Schnitt geometrischer Objekte}
\textbf{Erinnerung an Schulgeometrie}
\psset{xunit=1cm,yunit=1cm,runit=1cm}
\begin{pspicture}(0,-0.5)(10,4)
  \psaxes(0,0)(10,0)%
  \psaxes(0,0)(0,4)%
  \pscircle[fillcolor=black](3,1){0.1}%
  \pscircle(6,3){0.1}%
  \psline[linecolor=red](3,1)(6,3)%
  \only<1>{%
  \rput(2.5,1.5){$(x_{1},y_{1})$}%
  \rput(6,3.5){$(x_{2},y_{2})$}%
  }%
  \only<2-3>{%
  \psline[linecolor=green](3,1)(6,1)(6,3)%
  }%
  \only<2>{%
  \psbezier(3,0.9)(3,0.4)(4.5,1.1)(4.5,0.6)%
  \psbezier(4.5,0.6)(4.5,1.1)(6,0.4)(6,0.9)%
  \psbezier(6.1,1)(6.6,1)(5.9,2)(6.4,2)%
  \psbezier(6.4,2)(5.9,2)(6.6,3)(6.1,3)%
  \rput(4.5,0.4){$\Delta x$}%
  \rput(6.7,2){$\Delta y$}%
  }%
  \only<3>{%
  \psarc[linecolor=orange](3,1){1}{0}{33.69}%
  \rput(3.7,1.2){{\color{orange}$\varphi$}}%
  }%
\end{pspicture}

\begin{minipage}[5cm][3mm][s]{8cm}
\only<1>{%
Zwei Punkte mit Positionen $(x_{1},y_{1})$ und $(x_{2},y_{2})$.%
}%
\only<2>{%
Sei $\Delta x := x_{2}-x_{1}$ und $\Delta y := y_{2}-y_{1}$.
Dann beträgt Abstand der Punkte% $\sqrt{{\Delta x}^{2} + {\Delta y}^{2}}$.
}%
\only<3>{%
Der Winkel~$\varphi$ eines Segments zur $x$-Achse
lässt sich wie folgt errechnen:
$\varphi := arctan2({\Delta y},{\Delta x})$%
}%
\end{minipage}
\end{frame}

\begin{frame}
\frametitle{Kollision}
\framesubtitle{Schnitt geometrischer Objekte}
\textbf{Kreis mit Kreis}
\psset{xunit=1cm,yunit=1cm,runit=1cm}
\begin{pspicture}(0,-0.5)(10,4)
  \psaxes(0,0)(10,0)
  \psaxes(0,0)(0,4)
  \psline[linecolor=red](3,2)(6,3)
  \pscircle[linecolor=blue](3,2){1.5}
  \pscircle[linecolor=blue](6,3){1}
\end{pspicture}

Seien $P_{1}$ und $P_{2}$ die Mittelpunkte zweier Kreise
mit Radien $r_{1}$ und $r_{2}$.
Die Kreise schneiden sich genau dann, wenn
\[ abstand(P_{1}, P_{2}) < r_{1} + r_{2}. \]
\end{frame}

\begin{frame}
\frametitle{Kollision}
\framesubtitle{Schnitt geometrischer Objekte}
\begin{center}
{\LARGE Zu einfach?} \\
\vspace{0.4cm}
\only<2>{\LARGE Keine Sorge. Es wird besser.}
\end{center}
\end{frame}

\begin{frame}
\frametitle{Kollision}
\framesubtitle{Schnitt geometrischer Objekte}
\textbf{Konvexes Polygon mit konvexem Polygon}
\psset{xunit=1cm,yunit=1cm,runit=1cm}
\begin{pspicture}(0,-0.5)(10,4)
  \psaxes(0,0)(10,0)
  \psaxes(0,0)(0,4)
  \pspolygon[linecolor=blue](0.5,1.5)(1.5,0.5)(3,1)(3.5,2.5)(2,3.5)
  \pspolygon[linecolor=blue](3.5,2)(5,1)(7,1)(4.5,3.5)
  \only<2>{\psline[linecolor=red](2.8,0)(4.0,4)}
  \only<3>{\psline[linecolor=red](2.17,0)(4.83,4)}
  \only<4>{\psline[linecolor=red](4,4)(8,0)}
  \only<5>{\psline[linecolor=red](0,1)(10,1)}
  \only<6>{\psline[linecolor=red](0.5,4)(6.5,0)}
  \only<7>{\psline[linecolor=red](0,0.83)(2.375,4)}
  \only<8>{\psline[linecolor=red](1.25,4)(7.25,0)}
  \only<9>{\psline[linecolor=red](2.67,0)(4.0,4)}
%  \only<10>{\psline[linecolor=red](0,0)(10,3.33)}
%  \only<11>{\psline[linecolor=red](2,0)(0,2)}
\end{pspicture}

\end{frame}

\begin{frame}
\frametitle{Kollision}
\framesubtitle{Schnitt geometrischer Objekte}
\textbf{Bestimmung mittels Rotation}
\psset{xunit=1cm,yunit=1cm,runit=1cm}
\only<1,3>{\begin{pspicture}(0,-0.5)(10,4)}
\only<2>{\begin{pspicture}(0,-3.5)(10,1)}
\only<4>{\begin{pspicture}(0,-4)(10,0.5)}
  \psaxes(0,0)(10,0)
  \only<1,3>{\psaxes(0,0)(0,4)}
  \only<2>{\psaxes(0,0)(0,-4)(0,1)}
  \only<4>{\psaxes(0,0)(0,-5)(0,0)}
  \only<2>{\rput{-56.377}}{%
    \only<4>{\rput{-71.608}}{%
      \pspolygon[linecolor=blue](0.5,1.5)(1.5,0.5)(3,1)(3.5,2.5)(2,3.5)
      \pspolygon[linecolor=blue](3.5,2)(5,1)(7,1)(4.5,3.5)
      \only<1-2>{\psline[linecolor=red](2.17,0)(4.83,4)}
      \only<3-4>{\psline[linecolor=red](2.67,0)(4.0,4)}
    }%
  }%
\end{pspicture}

\end{frame}

\begin{frame}
\frametitle{Kollision}
\framesubtitle{Schnitt geometrischer Objekte}
\textbf{Konvexes Polygon mit Kreis}
\psset{xunit=1cm,yunit=1cm,runit=1cm}
\begin{pspicture}(0,-0.5)(10,4)
  \psaxes(0,0)(10,0)
  \psaxes(0,0)(0,4)
  \only<1>{\pscircle[linecolor=blue](2,2){1.2}}
  \only<2>{\pscircle[linecolor=blue](3,1.5){1.2}}
  \only<3>{\pscircle[linecolor=blue](3,2.5){1.2}}
  \pspolygon[linecolor=blue](3.5,1)(5,0.5)(7,1)(4.5,3.5)
\end{pspicture}

\end{frame}

\end{document}
