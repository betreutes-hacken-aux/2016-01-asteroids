\documentclass{article}

\usepackage[utf8]{inputenc}
\usepackage[ngerman]{babel}

\title{Asteroids}
\date{23. Juni 2016}

\begin{document}
\maketitle

\section{Gameloop}

\section{Movement}
\subsection{Homogene Koordinaten}
Der Vorteil von homogenen Koordinaten besteht darin,
dass sich eine beliebige Verkettung
aufeinanderfolgender Translationen und Rotationen
zu einer Matrix zusammengefasst werden kann.
\subsubsection{Karthesische Koordinaten zu homogenen Koordinaten}
\begin{description}
 \item[Ortsvektoren:]
  \begin{equation}
   \left(\begin{array}{c}
    x \\
    y \\
   \end{array}\right)
   \rightarrow
   \left(\begin{array}{c}
    x \\
    y \\
    1 \\
   \end{array}\right)
  \end{equation}

 \item[Richtungsvektoren:]
  \begin{equation}
   \left(\begin{array}{c}
    x \\
    y \\
   \end{array}\right)
   \rightarrow
   \left(\begin{array}{c}
    x \\
    y \\
    0 \\
   \end{array}\right)
  \end{equation}
\end{description}

\subsubsection{Rotation und Translation}
\begin{description}
 \item[Rotationsmatriz]
  für Rotation um $\varphi$ entgegen dem Uhrzeigersinn:
  \begin{equation}
   \left(\begin{array}{ccc}
    cos(\varphi) & -sin(\varphi) & 0 \\
    sin(\varphi) & cos(\varphi) & 0 \\
    0 & 0 & 1 \\
   \end{array}\right)
  \end{equation}
  Anwendung:
  \begin{equation}
   \left(\begin{array}{ccc}
    cos(\varphi) & -sin(\varphi) & 0 \\
    sin(\varphi) & cos(\varphi) & 0 \\
    0 & 0 & 1 \\
   \end{array}\right)
   \cdot
   \left(\begin{array}{ccc}
    x \\
    y \\
    w \\
   \end{array}\right)
   =
   \left(\begin{array}{ccc}
    x \cdot cos(\varphi) - y \cdot sin(\varphi) \\
    x \cdot sin(\varphi) + y \cdot cos(\varphi) \\
    w \\
   \end{array}\right)
  \end{equation}
  Beispiel: Den Punkt $(2, 1)^{T}$ um $90^{\circ}$ rotieren.
  \begin{equation}
   \left(\begin{array}{ccc}
    cos(90^{\circ}) & -sin(90^{\circ}) & 0 \\
    sin(90^{\circ}) & cos(90^{\circ}) & 0 \\
    0 & 0 & 1 \\
   \end{array}\right)
   \cdot
   \left(\begin{array}{ccc}
    2 \\
    1 \\
    1 \\
   \end{array}\right)
   =
   \left(\begin{array}{ccc}
    -1 \\
    2 \\
    1 \\
   \end{array}\right)
  \end{equation}

 \item[Translation:]
  \begin{equation}
   \left(\begin{array}{ccc}
    1 & 0 & t_{x} \\
    0 & 1 & t_{y} \\
    0 & 0 & 1 \\
   \end{array}\right)
  \end{equation}
  Anwendung auf einen Richtungsvektor:
  \begin{equation}
   \left(\begin{array}{ccc}
    1 & 0 & t_{x} \\
    0 & 1 & t_{y} \\
    0 & 0 & 1 \\
   \end{array}\right)
   \cdot
   \left(\begin{array}{ccc}
    x \\
    y \\
    1 \\
   \end{array}\right)
   =
   \left(\begin{array}{ccc}
    x + t_{x} \cdot w \\
    y + t_{y} \cdot w \\
    1 \\
   \end{array}\right)
  \end{equation}
  Beobachtung:
  Eine Translation wirkt nur auf Ortsvektoren,
  nicht auf Richtungsvektoren.
\end{description}

\subsection{OpenGL 2.x}
\texttt{glRotate}
\texttt{glTranslate}

\subsection{OpenGL 4.x}


\section{Pew! Pew! Pew!}

\section{Schnitterkennung}
\subsection{Kreis mit Kreis}
Ein Kreis lässt sich über seine Position~$(x, y)^{T}$
und seinen Radius~$r$ spezifizieren.
Zwei Kreise mit Positionen~$(x_{1}, y_{1})^{T}$ und~$(x_{2}, y_{2})^{T}$
und Radien~$r_{1}$ und~$r_{2}$ überschneiden sich genau dann,
wenn $\sqrt{(x_{1} - x_{2})^{2} + (y_{1} - y_{2})^{2}} < r_{1} + r_{2}$.

\subsection{Konvexes Polygon mit konvexem Polygon}
Zwei konvexe Polygone~$P_{1}$ und~$P_{2}$ überschneiden sich genau dann,
wenn es keine Trennlinie gibt.

\subsection{konvexem Polygon mit Kreis}

\bibliography{referenzen.bib}

\end{document}
